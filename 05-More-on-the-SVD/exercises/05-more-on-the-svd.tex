 % --------------------------------------------------------------
% Based on a homework template by Dana Ernst (MTH320 Homework
% template on Overleaf).
% --------------------------------------------------------------

\documentclass[12pt]{article}

\usepackage{graphicx}
\graphicspath{{./images/}}
\usepackage[margin=1in]{geometry} 
\usepackage{amsmath,amsthm,amssymb}
% https://tex.stackexchange.com/questions/146306/how-to-make-horizontal-lists
\usepackage[inline]{enumitem} % allows using letters in enumerate list environment
\usepackage[mathscr]{euscript}
% source: https://stackoverflow.com/questions/3175105/inserting-code-in-this-latex-document-with-indentation

\usepackage{listings}
\usepackage{color}
\usepackage{hyperref}

\newtheorem{theorem}{Theorem}

\definecolor{dkgreen}{rgb}{0,0.6,0}
\definecolor{gray}{rgb}{0.5,0.5,0.5}
\definecolor{mauve}{rgb}{0.58,0,0.82}

\lstset{frame=tb,
	language=C, % language for code listing
	aboveskip=3mm,
	belowskip=3mm,
	showstringspaces=false,
	columns=flexible,
	basicstyle={\small\ttfamily},
	numbers=none,
	numberstyle=\tiny\color{gray},
	keywordstyle=\color{blue},
	commentstyle=\color{dkgreen},
	stringstyle=\color{mauve},
	breaklines=true,
	breakatwhitespace=true,
	tabsize=4
}

\newcommand{\N}{\mathbb{N}}
\newcommand{\Z}{\mathbb{Z}}

\newenvironment{ex}[2][Exercise]{\begin{trivlist}
		\item[\hskip \labelsep {\bfseries #1}\hskip \labelsep {\bfseries #2.}]}{\end{trivlist}}

\newenvironment{sol}[1][Solution]{\begin{trivlist}
		\item[\hskip \labelsep {\bfseries #1:}]}{\end{trivlist}}


\begin{document}

% --------------------------------------------------------------
%                         Start here
% --------------------------------------------------------------

\noindent Sergio Garcia Tapia \hfill

\noindent{\small Numerical Linear Algebra, Lloyd Trefethen and David Bau III} \hfill 

\noindent{\small Lecture 5: More on the SVD} \hfill 

\noindent\today
\section*{Lecture 5: More on the SVD}

\begin{ex}{1}
	In Example 3.1, we considered the matrix (3.7) and asserted, among other things, that its 2-norm is
	approximately $2.9208$. Using the SVD, work out (on paper) the exact values of $\sigma_{\text{min}}(A)$ and
	$\sigma_{\text{max}}(A)$.
\end{ex}

\begin{sol}
	\
	The matrix in question is
	\[
	A = \begin{bmatrix}
		1 & 2\\
		0 & 2
	\end{bmatrix}
	\]
	By Theorem 5.4, the nonzero singular values of $A$ are the square roots of the eigenvalues of $A^*A$ and $AA^*$
	(these matrices have the same eigenvalues). Thus we proceed to find eigenvalues of $A^*A$
	\begin{align*}
		A^*A = 
		\begin{bmatrix}
			1 & 0\\
			2 & 2
		\end{bmatrix}
		\begin{bmatrix}
			1 & 2\\
			0 & 2
		\end{bmatrix}
		=\begin{bmatrix}
			1 & 2\\
			2 & 8
		\end{bmatrix}
	\end{align*}
	Now we find its eigenvalues which will be the values $\lambda$ such that $A^*A-\lambda I$ is singular.
	That is, we want $(A^*A-\lambda I)v=0$ for nonzero $v\in\mathbb{C}^2$:
	\begin{align*}
		0=(A^*A-\lambda I)v=\begin{bmatrix}
			1-\lambda & 2\\
			2 & 8-\lambda
		\end{bmatrix}
		\begin{bmatrix}
			x\\
			y
		\end{bmatrix}
	\end{align*}
	This is equivalent to the system:
	\begin{align*}
		0 &= (1-\lambda)x + 2y\\
		0 &= 2x + (8-\lambda)y
	\end{align*}
	We can multiply the second equation by $-\frac{1-\lambda}{2}$ and add it to the first to get:
	\begin{align*}
		0 &= 0x +\left(2-\frac{1}{2}(1-\lambda)(8-\lambda)\right)y\\
		0 &= 2x + (8-\lambda)y
	\end{align*}
	This will have nonzero solutions if $2-\frac{1}{2}(1-\lambda)(8-\lambda)=0$, which is equivalent
	to the quadratic equation $\lambda^2-9\lambda+4=0$. The solutions can be obtained using the quadratic
	formula, and they are
	\begin{align*}
		\lambda_{\pm} &= \frac{9\pm \sqrt{65}}{2}
	\end{align*}
	Both are positive, as expected because $A^*A$ is invertible so it has nonzero eigenvalues, and it is
	also a positive semidefinite operator, so its eigenvalues must be nonnegative. Now
	\[
	\sigma_{\text{min}} = \sqrt{\frac{9-\sqrt{65}}{2}},\quad \sigma_{\text{max}}=\sqrt{\frac{9+\sqrt{65}}{2}}
	\approx 2.9208
	\]
\end{sol}

\begin{ex}{5.2}
	Using the SVD prove that any matrix in $\mathbb{C}^{m\times n}$ is the limit of a sequence of matrices of
	full rank. In other words, prove that the set of full-rank matrices is a dense subset of $\mathbb{C}^{m\times n}$.
	Using the 2-norm for your proof. (The norm doesn't matter, since all norms on a finite-dimensional
	space are equivalent).
\end{ex}

\begin{sol}
	\
	\begin{proof}
		Let $A\in\mathbb{C}^{m\times n}$. 
	\end{proof}
\end{sol}

\begin{ex}{5.3}
	Consider the matrix
	\[
	A=\begin{bmatrix}
		-2 & 11\\
		-10 & 5
	\end{bmatrix}
	\]
	\begin{enumerate}
		\item Determine, on paper, a real SVD of $A$ in the form $A=U\Sigma V^T$. The SVD is not unique, so
		find one that has the minimal number of minus signs in $U$ and $V$.
		\item List the singular values, left singular vectors, and right singular vectors of $A$. Draw a
		careful, labeled picture of the unit ball in $\mathbb{R}^2$ and its image under $A$, together with
		the singular vectors, with the coordinates of their vertices marked.
		\item What are the 1-, 2-, $\infty$- and Frobenius norms of $A$?
		\item Find $A^{-1}$ not directly, but via the SVD.
		\item Find the eigenvalues $\lambda_1$, $\lambda_2$ of $A$.
		\item Verify that $\det A=\lambda_1\lambda_2$ and $|\det A|=\sigma_1\sigma_2$.
		\item What is the area of the ellipsoid into which $A$ maps the unit ball of $\mathbb{R}^2$?
	\end{enumerate}
\end{ex}

\begin{sol}
	\
	\begin{enumerate}[label=(\alph*)]
		\item The matrix $A$ is nonsingular and thus its singular values are all nonzero. By Theorem 5.4,
		the nonzero singular values of $A$ are the square roots of the nonzero eigenvalues of $A^*A$ or $AA^*$.
		Moreover, suppose that $A=U\Sigma V^*$, so that $A^*A=V\Sigma^2V^*$. Then the eigenvectors of
		$A^*A$ are the right singular vectors of $A$. Similarly, the eigenvectors of $AA^*$ are the left
		singular vectors of $A$. Thus we begin by finding the eigenvalues of $A^*A$:
		\begin{align*}
			A^*A =
			\begin{bmatrix}
				-2 & -10\\
				11 & 5
			\end{bmatrix}
			\begin{bmatrix}
				-2 & 11\\
				-10 & 5
			\end{bmatrix}
			=
			\begin{bmatrix}
				104 & -72\\
				-72 & 146
			\end{bmatrix}
		\end{align*}
		If $\lambda$ is an eigenvalue of $A^*A$, then $A^*A-\lambda I$ is singular, which means that
		$(A^*A-\lambda I)v=0$ has a nonzero solution $v\in\mathbb{C}^2$:
		\begin{align*}
			0 =(A^*v-\lambda I)v= \begin{bmatrix}
				104-\lambda & -72\\
				-72 & 146-\lambda
			\end{bmatrix}
			\begin{bmatrix}
				x\\
				y
			\end{bmatrix}
		\end{align*}
		Hence we have the system
		\begin{align*}
			0&=(104-\lambda)x -72y\\
			0 &= -72x + (146-\lambda)y
		\end{align*}
		We can multiply the second equation by $\frac{1}{72}(104-\lambda)$ and add it to the first to
		get
		\begin{align*}
			0 &= 0x +\left(\frac{1}{72}(104-\lambda)(146-\lambda)-72\right)y \\
			0 &= -72x + (146-\lambda)y
		\end{align*}
		This equation will have nonzero solutions if
		\begin{align*}
			\frac{1}{72}(104-\lambda)(146-\lambda)-72&=0
		\end{align*}
		which simplifies to
		\begin{align*}
			\lambda^2-250\lambda +10000&=0
		\end{align*}
		The solutions are
		\begin{align*}
			\sigma_{1}^2 = 200,\quad \sigma_2^2 = 50
		\end{align*}
		Thus $\sigma_1 =\sqrt{200}=10\sqrt{2}$, and $\sigma_2==\sqrt{50}=5\sqrt{2}$.
		Now to find the eigenvectors, we substitute $\sigma_1^2$ and $\sigma_2^2$ in our equation
		\begin{align*}
			0 &= 0x +\left(\frac{1}{72}(104-\lambda)(146-\lambda)-72\right)y \\
			0 &= -72x + (146-\lambda)y
		\end{align*}
		Since $\sigma_1^2$ and $\sigma_2^@$ both make the coefficient of $y$ in the first equation 0,
		this implies that $72x=(146-\lambda_j)y$. If we let $y=72$, then
		\begin{align*}
			\sigma_1^2&=200\to \begin{bmatrix}
				-54\\
				72
			\end{bmatrix}
			\quad\implies\quad v_1=\frac{1}{90}
			\begin{bmatrix}
				-54\\
				72
			\end{bmatrix}
			=\frac{1}{5}\begin{bmatrix}
				-3\\
				4
			\end{bmatrix}
			\\
			\sigma_2^2&=50\to
			\begin{bmatrix}
				96\\
				72
			\end{bmatrix}
			\quad \implies\quad 
			v_2=
			\frac{1}{120}
			\begin{bmatrix}
				96\\
				72
			\end{bmatrix}
			=\frac{1}{5}\begin{bmatrix}
				4 \\
				3
			\end{bmatrix}
		\end{align*}
		Here $v_1,v_2$ is an orthonormal list. Next we find the eigenvectors of $AA^*$, using the fact that
		the eigenvalues are the same as those of $A^*A$. We begin by computing $AA^*$:
		\begin{align*}
			AA^*=
			\begin{bmatrix}
				-2 & 11\\
				-10 & 5
			\end{bmatrix}
			\begin{bmatrix}
				-2 & -10\\
				11 & 5
			\end{bmatrix}
			=
			\begin{bmatrix}
				125 & 75\\
				75 &  125
			\end{bmatrix}
		\end{align*}
		Now we find eigenvectors by finding the nonzero solutions to:
		\begin{align*}
			0&=(AA^*-\sigma_1^2 I)v=\begin{bmatrix}
				-75 & 75\\
				75 & -75
			\end{bmatrix}\begin{bmatrix}
			x\\
			y
		\end{bmatrix}\\
		0&=(AA^*-\sigma_2^2 I)v=\begin{bmatrix}
			75 & 75\\
			75 & 75
		\end{bmatrix}\begin{bmatrix}
			x\\
			y
		\end{bmatrix}
		\end{align*}
		The first system has the nontrivial solution $x=y=1$, and the second system has the nontrivial solution
		$x=1$ and $y=-1$. Thus
		\begin{align*}
			u_1 &=\frac{1}{\sqrt{2}} \begin{bmatrix}
				1\\
				1
			\end{bmatrix}
			\quad
			u_2 = \frac{1}{\sqrt{2}}\begin{bmatrix}
				1 \\
				-1
			\end{bmatrix}
		\end{align*}
		is an orthonormal list of eigenvectors of $AA^*$. Now we let $U$ be the matrix whose columns are $u_1$ and $u_2$,
		let $V$ be the matrix whose columns are $v_1$ and $v_2$, and let $\Sigma$ be the matrix whose diagonal
		entries are $\sigma_1=\sqrt{200}$ and $\sigma_2=\sqrt{50}$, then $U\Sigma V^*$ is an SVD of $A$. To see
		this, we can multiply them
		\begin{align*}
			U\Sigma V^*&=\frac{1}{\sqrt{2}}\begin{bmatrix}
				1 & 1\\
				1 & -1
			\end{bmatrix}
			\begin{bmatrix}
				\sqrt{200} & 0\\
				0 & \sqrt{50}
			\end{bmatrix}
			\cdot \frac{1}{5}\begin{bmatrix}
				-3 & 4\\
				4 & 3
			\end{bmatrix}\\
		&=\begin{bmatrix}
			1 & 1\\
			1 & -1
		\end{bmatrix}
		\begin{bmatrix}
			2 & 0\\
			0 & 1
		\end{bmatrix}
		\begin{bmatrix}
			-3 & 4\\
			4 & 3
		\end{bmatrix}\\
		&=A
		\end{align*}
		\item The singular value of $A$ are
		\[
		\sigma_1=\sqrt{200},\quad \sigma_2=\sqrt{50}
		\]
		The left singular vectors are
		\begin{align*}
			u_1&=\frac{1}{\sqrt{2}}
			\begin{bmatrix}
				1\\
				1
			\end{bmatrix}
			\quad 
			u_2=\frac{1}{\sqrt{2}}
			\begin{bmatrix}
				1\\
				-1
			\end{bmatrix}
		\end{align*}
		The right singular vectors are
		\begin{align*}
			v_1&=\frac{1}{5}
			\begin{bmatrix}
				-3\\
				4
			\end{bmatrix}
			\quad 
			v_2=\frac{1}{5}
			\begin{bmatrix}
				4\\
				3
			\end{bmatrix}
		\end{align*}
		\item The 1-norm of $A$ is the largest 1-norm among its columns, by Example 3.3 (Equation 3.9).
		Since $\lVert a_1\rVert_1 = |-2| + |-10| = 12$ and $\lVert a_2\rVert_1 = 11+5=17$, we see that
		$\lVert A\rVert_1=17$. By Example 3.4, the infinity norm of $A$ is the largest 1-norm of its rows,
		which is the largest of $|-10|+5$ and $|-2|+11$, and hence $\lVert A\rVert_\infty = 15$.
		By Theorem 5.3, $\lVert A\rVert_2=\sigma_1=\sqrt{200}$, the largest singular value of $A$.
		By the same Theorem, $\lVert A\rVert_F=\sqrt{\sigma_1^2+\sigma_2^2}=\sqrt{200+50}=\sqrt{250}$.
		\item To find $A^{-1}$ we note that $U$ and $V^T$ are unitary and $\Sigma$ is diagonal, so
		$A^{-1}=(V^*)^{-1}\Sigma^{-1}U^{-1}$, and hence $A^{-1}=V\Sigma^{-1}U^T$
		\begin{align*}
			A^{-1}&=\frac{1}{\sqrt{2}}\begin{bmatrix}
				1 & 1\\
				1 & -1
			\end{bmatrix}
			\begin{bmatrix}
				\frac{1}{\sqrt{200}} & 0\\
				0 & \frac{1}{\sqrt{50}}
			\end{bmatrix}
			\frac{1}{5}
			\begin{bmatrix}
				-3 & 4\\
				4 & 3
			\end{bmatrix}\\
			&=\begin{bmatrix}
				1 & 1\\
				1 & -1
			\end{bmatrix}
			\begin{bmatrix}
				\frac{1}{100} & 0\\
				0 & \frac{1}{50}
			\end{bmatrix}
			\begin{bmatrix}
				-3 & 4\\
				4 & 3
			\end{bmatrix}\\
			&=\frac{1}{100}
			\begin{bmatrix}
				1 & 1\\
				1 & -1
			\end{bmatrix}
			\begin{bmatrix}
				1 & 0\\
				0 & 2
			\end{bmatrix}
			\begin{bmatrix}
				-3 & 4\\
				4 & 3
			\end{bmatrix}\\
			&=\frac{1}{100}\begin{bmatrix}
				5 & -11\\
				10 & -2
			\end{bmatrix}
		\end{align*}
		\item The eigenvalues of $A$ are the values that make $A-\lambda I$ a singular matrix. We want
		nonzero solutions to
		\begin{align*}
			(A-\lambda I)v=\begin{bmatrix}
				-2-\lambda& 11\\
				-10 & 5-\lambda
			\end{bmatrix}
			\begin{bmatrix}
				x\\
				y
			\end{bmatrix}
		\end{align*}
		We have the equations
		\begin{align*}
			0&=(-2-\lambda)x+11y\\
			0&=-10x+(5-\lambda)y
		\end{align*}
		Multiplying the second equation by $\frac{1}{10}(-2-\lambda)$ and adding it to the first equation,
		we get
		\begin{align*}
			0&=0x + \left(\frac{1}{10}(-2-\lambda)(5-\lambda)+11\right)y\\
			0&=-10x + (5-\lambda)y
		\end{align*}
		The system will admit nontrivial solutions if
		\begin{align*}
			\frac{1}{10}(-2-\lambda)(5-\lambda)+11&=0\\
			(-2-\lambda)(5-\lambda)+110&=0\\
			-10-3\lambda+\lambda^2+110&=0\\
			\lambda^2-3\lambda +100&=0\\
		\end{align*}
		which has solutions
		\begin{align*}
			\lambda_1&=\frac{3+i\sqrt{391}}{2},\quad \lambda_2=\frac{3-i\sqrt{391}}{2}
		\end{align*}
		Hence the eigenvalues are $\lambda_1=15$ and $\lambda_2=-8$.
		\item $\det A=a_{11}\cdot a_{22}-a_{21}\cdot a_{12}=-2\cdot 5 - (-10)\cdot 11=-10+110 = 100$
		This also equals $\lambda_1\cdot \lambda_2$ and $\sigma_1\cdot \sigma_2$.
		\item The area of an ellipse is $A=\pi ab$, where $a$ and $b$ are the lengths of the ellipse axes.
		Thus $A=\pi \cdot\sigma_1\cdot \sigma_2=\pi \cdot \sqrt{200}\cdot \sqrt{50}=100\pi$.
	\end{enumerate}
\end{sol}

\begin{ex}{5.4}
	Suppose $A\in\mathbb{C}^{m\times m}$ has an SVD $A=U\Sigma V^*$. Find an eigenvalue decomposition of the
	$2m\times 2m$ hermitian matrix
	\begin{align*}
		\begin{bmatrix}
			0 & A^*\\
			A & 0
		\end{bmatrix}
	\end{align*}
\end{ex}

\begin{sol}
	\
	Let $x,y\in\mathbb{C}^m$ and set
	\[
	B=\begin{bmatrix}
		0 & A^*\\
		A & 0
	\end{bmatrix}
	\quad
	z=\begin{bmatrix}
		x\\
		y
	\end{bmatrix}
	\]
	That is, $z$ is the vector in $\mathbb{C}^{2m}$ with $x$ stacked on top of $y$ If $\lambda$ is an eigenvalue of $B$,
	then
	\begin{align*}
		Bz&=\lambda z\\
		\begin{bmatrix}
			0 & A^*\\
			A & 0
		\end{bmatrix}
		\begin{bmatrix}
			x\\
			y
		\end{bmatrix}
		&=
		\lambda \begin{bmatrix}
			x\\
			y
		\end{bmatrix}\\
		\begin{bmatrix}
			A^*y\\
			Ax
		\end{bmatrix}
		&=
		\lambda \begin{bmatrix}
			x\\
			y
		\end{bmatrix}
	\end{align*}
	The equations $A^*y=\lambda x$ and $Ax=\lambda y$ reduce to $A^*Ax=\lambda^2x$ and $AA^*y=\lambda^2y$.
	In other words, every eigenvalue of $A^*A$ is the square of an eigenvalue of $B$. Thus the eigenvalues
	are $\sigma_1,-\sigma_1,\ldots,\sigma_n,-\sigma_n$. In particular, the eigenvectors of $B$ are the
	left and right singular vectors stacked on top of each other, of the form
	\begin{align*}
		\begin{bmatrix}
			v_j\\
			u_j
		\end{bmatrix}
		\quad
		\text{and}
		\quad 
		\begin{bmatrix}
			v_j\\
			-u_j
		\end{bmatrix}
	\end{align*}
	where $j\in\{1,\ldots,m\}$. In particular,
	\begin{align*}
		\begin{bmatrix}
			0 & A^*\\
			A & 0
		\end{bmatrix}
		\begin{bmatrix}
			v_j\\
			u_j
		\end{bmatrix}
		&=\begin{bmatrix}
			A^*u_j\\
			Av_j
		\end{bmatrix}
		=\sigma_1\begin{bmatrix}
			v_j\\
			u_j
		\end{bmatrix}\\
		\begin{bmatrix}
			0 & A^*\\
			A & 0
		\end{bmatrix}
		\begin{bmatrix}
			v_j\\
			-u_j
		\end{bmatrix}
		&=\begin{bmatrix}
			-A^*u_j\\
			Av_j
		\end{bmatrix}
		=-\sigma_1\begin{bmatrix}
			v_j\\
			-u_j
		\end{bmatrix}
	\end{align*}
\end{sol}

\end{document}