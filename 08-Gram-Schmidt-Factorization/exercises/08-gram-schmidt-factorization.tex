 % --------------------------------------------------------------
% Based on a homework template by Dana Ernst (MTH320 Homework
% template on Overleaf).
% --------------------------------------------------------------

\documentclass[12pt]{article}

\usepackage{graphicx}
\graphicspath{{./images/}}
\usepackage[margin=1in]{geometry} 
\usepackage{amsmath,amsthm,amssymb}
\usepackage{mathdots}
% https://tex.stackexchange.com/questions/146306/how-to-make-horizontal-lists
\usepackage[inline]{enumitem} % allows using letters in enumerate list environment
\usepackage[mathscr]{euscript}
% source: https://stackoverflow.com/questions/3175105/inserting-code-in-this-latex-document-with-indentation

\usepackage{listings}
\usepackage{color}
\usepackage{hyperref}

\newtheorem{theorem}{Theorem}

\definecolor{dkgreen}{rgb}{0,0.6,0}
\definecolor{gray}{rgb}{0.5,0.5,0.5}
\definecolor{mauve}{rgb}{0.58,0,0.82}

\lstset{frame=tb,
	language=C, % language for code listing
	aboveskip=3mm,
	belowskip=3mm,
	showstringspaces=false,
	columns=flexible,
	basicstyle={\small\ttfamily},
	numbers=none,
	numberstyle=\tiny\color{gray},
	keywordstyle=\color{blue},
	commentstyle=\color{dkgreen},
	stringstyle=\color{mauve},
	breaklines=true,
	breakatwhitespace=true,
	tabsize=4
}

\newcommand{\N}{\mathbb{N}}
\newcommand{\Z}{\mathbb{Z}}

\newenvironment{ex}[2][Exercise]{\begin{trivlist}
		\item[\hskip \labelsep {\bfseries #1}\hskip \labelsep {\bfseries #2.}]}{\end{trivlist}}

\newenvironment{sol}[1][Solution]{\begin{trivlist}
		\item[\hskip \labelsep {\bfseries #1:}]}{\end{trivlist}}


\begin{document}

% --------------------------------------------------------------
%                         Start here
% --------------------------------------------------------------

\noindent Sergio Garcia Tapia \hfill

\noindent{\small Numerical Linear Algebra, Lloyd Trefethen and David Bau III} \hfill 

\noindent{\small Lecture 8: Gram Schmidt Factorization} \hfill 

\noindent\today
\section*{Lecture 8: Gram Schmidt Factorization}

\begin{ex}{1}
	Let $A$ be an $m\times n$ matrix. Determine the exact numbers of floating point additions, subtractions, and
	multiplications involved in computing the factorization $A=\hat{Q}\hat{R}$ by Algorithm 8.1.
\end{ex}

\begin{sol}
	\
	The presented algorithm was:
	\begin{lstlisting}[language={}]
# Initialize v_i to columns of A
for i = 1 to n
	v_i = a_i

# Create orthogonal list from columns of A
for i = 1 to n
	r_ii = ||v_i|| # Norm of v_i
	q_i = v_i/r_ii # Normalize
	for j = i + 1 to n
		r_ij = adj(q_i) * v_j # inner product of q_i and v_j
		v_j = v_j - r_ijq_i

	\end{lstlisting}
	The norm operation,$\lVert v_i\rVert$, involves $m$ products (squaring each entry), $m-1$ addition (to add
	the $m$ products), and one square root. The expression $q_i=v_i/r_{ii}$ involves $m$ divisions.
	Each of these two operations occurs exactly $n$ times, since they occur in the outer loop.
	
	\
	As discussed in page 59, the operations of the inner loop are
	\begin{align*}
	r_{ij} &= q_i^*v_j\\
	v_j &= v_j - r_{ij}q_i
	\end{align*}
	Since $a_i\in\mathbb{C}^{m}$, the inner product $q_i^*v_j$ involves $m$ multiplications (one for
	each pair of entries) and $m-1$ additions (one for each pair of the $m$ multiplication results).
	Meanwhile, $r_{ij}q_i$ involves $m$ multiplications, and subtracting the result from $v_j$ in the
	expression $v_j-v_j-r_{ij}q_i$ involves $m$ subtractions. Thus the exact number of operations in
	each iteration of the inner loop is $4m-1$: $2m$ multiplications, $m$ subtractions, and $m-1$ additions.
	
	\
	Note that each inner loop operations occurs a constant number of times, so its helpful to compute the
	sum:
	\begin{align*}
	\sum_{i=1}^{n}\sum_{j=i+1}^{n}(1)&=\sum_{i=1}^{n}(n-i)\\
	&=\sum_{i=1}^{n}(n)-\sum_{i=1}^{n}i\\
	&=n^2 - \frac{n(n+1)}{2}\\
	&=\frac{2n^2}{2}-\frac{n^2+n}{2}\\
	&=\frac{n(n-1)}{2}
	\end{align*}
	\
	Altogether:
	\begin{align*}
		\text{Square Roots} &: 1\\
		\text{Divisions} &: n\\
		\text{Addition} &: n\cdot (m-1) + \sum_{i=1}^{n}\sum_{j=i+1}^{n}(m-1)\\
		&=n(m-1)+\frac{(m-1)n(n-1)}{2}\cdot\\
		&=\frac{(m-1)n(n+1)}{2}\\
		\text{Subtractions} &: \sum_{j=i+1}^{n}(m)=\frac{mn(n-1)}{2}\\
		\text{Multiplications} &: m + \sum_{j=i+1}^{n}(2m)=m+mn(n-1)\\
		&=m(n^2-n+1)
	\end{align*}
\end{sol}

\begin{ex}{8.2}
	Write a MATLAB function \texttt{[Q,R] = mgs(A)} (see next lecture) that computes a reduced QR factorization
	$A=\hat{Q}\hat{R}$ of an $m\times n$ matrix with $m\geq n$ using modified Gram Schmidt orthogonalization.
	The output variables are a matrix $Q\in\mathbb{C}^{m\times n}$ with orthonormal columns and a triangular
	matrix $R\in\mathbb{C}^{m\times n}$.
\end{ex}

\begin{ex}{3}
	Each upper-triangular matrix $R_j$ of p. 61 can be interpreted as the product of a diagonal matrix and
	a unit upper-triangular matrix (i.e., an upper-triangular matrix with 1 on the diagonal). Explain exactly
	what these factors are, and which line of Algorithm 8.1 corresponds to each.
\end{ex}

\begin{sol}
	\
	Consider the example matrix given in page 61 for $R_1\in\mathbb{C}^{n\times n}$:
	\[
	R_1=\begin{bmatrix}
		\frac{1}{r_{11}} & \frac{-r_{12}}{r_{11}} & \frac{-r_{13}}{r_{11}} & \cdots\\
		& 1 & {} & \\
		{} & {} & 1 & {}\\
		{} & {} & {} & \ddots
	\end{bmatrix}
	\]
	Then $R_1=D_1T_1$, where $D_1,T_1\in\mathbb{C}^{n\times n}$, $D_1$ is a diagonal matrix, and $T_1$ is
	an upper triangular matrix of the form
	\[
	D_1=\begin{bmatrix}
		\frac{1}{r_{11}} & 0 & \cdots & 0\\
		0 & 1 & 0 & \vdots\\
		\vdots & 0 & \ddots & 0\\
		0 & \cdots& {} & 1
	\end{bmatrix}
	\quad\quad
	T_1=\begin{bmatrix}
		1 & -r_{12} & -r_{13} & \cdots\\
		0 & 1 & 0 & \cdots\\
		0 & 0 & \ddots & 0
	\end{bmatrix}
	\]
	That is, $D_{1}$ has is diagonal with entry every 1, except the first diagonal entry where $d_{11}=\frac{1}{r_{11}}$.
	Then $T_1$ is upper-triangular with all diagonal entries 1. Every other entry is 0, except the first row,
	where $t_{11}=1$ and $t_{1j}=-r_{1j}$ for $j>1$. In general, $D_i,T_i\in\mathbb{C}^{n\times n}$
	satisfy $R_i=D_iT_i$, where
	\begin{align*}
		d_{kj}=\begin{cases}
			\frac{1}{r_{ii}} & \text{if }k=i\text{ and } j=i\\
			1 & \text{if }k=i \text{ and } j\neq i\\
			0 & \text{Otherwise}
		\end{cases}
		\quad\quad
		t_{kj} = \begin{cases}
			1 & \text{if } k = i\\
			0 & \text{if } k \neq i\\
			r_{kj} & \text{if } k = i \text{and } j > i
		\end{cases}
	\end{align*}
	In Algorithm 8.1, the $D_i$ matrix corresponds to the step $q_i=v_i/r_{ii}$, which normalizes $v_i$.
	The $T_i$ matrix corresponds to the inner loop
	\begin{align*}
	r_{ij}&=q_i^*v_j\\
	v_j&=v_j-r_{ij}q_i
	\end{align*}
	where $r_{ij}$ is computed and then multiplied by the outer loop column before it is subtracted.
\end{sol}

\end{document}